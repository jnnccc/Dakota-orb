\begin{DoxyAuthor}{Author}
Michael S. Eldred, William J. Bohnhoff, Richard V. Field, Jr.
\end{DoxyAuthor}
\hypertarget{index_DevIntro}{}\section{Introduction}\label{index_DevIntro}
The P\+E\+C\+OS (Parallel Environment for Creation Of Stochastics) library initiated as a capability to generate samples of random fields (R\+Fs) and stochastic processes (S\+Ps) from a set of user-\/defined power spectral densities (P\+S\+Ds). The R\+F/\+SP may be either Gaussian or non-\/\+Gaussian and either stationary or nonstationary, and the resulting sample is intended for run-\/time query by parallel finite element simulation codes.

P\+E\+C\+OS has grown to include univariate and multivariate polynomial approximations, using either orthogonal or interpolation polynomials, along with numerical integration drivers for computing expansion coefficients.

The Developers Manual documents the class structures used by the P\+E\+C\+OS library. It derives directly from annotation of the source code.\hypertarget{index_DevOverview}{}\section{Overview of P\+E\+C\+OS}\label{index_DevOverview}
The P\+E\+C\+OS library collects a variety of utilities for transformation of probability distributions, transformations between frequency and time domain data, and modeling using basis functions.

The use of class hierarchies provides a mechanism for extensibility in P\+E\+C\+OS components. In each of the various class hierarchies, adding a new capability typically involves deriving a new class and providing a few virtual function redefinitions. These redefinitions define the coding portions specific to the new derived class, with the common portions already defined at the base class. Thus, with a relatively small amount of new code, the existing facilities can be extended, reused, and leveraged for new purposes.

The software components are presented in the following sections using a top-\/down order.\hypertarget{index_DevDataTrans}{}\subsection{Data Transformations}\label{index_DevDataTrans}
Class hierarchy\+: \hyperlink{classPecos_1_1DataTransformation}{Data\+Transformation}.

Data transformations provide a mechanism for moving data sets between the frequency domain and the time domain. These transformations typically involve modeling through the use of scalar random variables (transformed to standardized probability distributions) and a set of spatial and/or temporal basis functions. Thus the Basis\+Function and \hyperlink{classPecos_1_1ProbabilityTransformation}{Probability\+Transformation} capabilities are components of the \hyperlink{classPecos_1_1DataTransformation}{Data\+Transformation} capability. Specific data transformations include\+:


\begin{DoxyItemize}
\item Forward\+Transformation\+: from time domain to frequency domain based on F\+FT. A placeholder for now.


\item \hyperlink{classPecos_1_1InverseTransformation}{Inverse\+Transformation}\+: from frequency domain (P\+SD) to time domain. 
\begin{DoxyItemize}
\item \hyperlink{classPecos_1_1FourierInverseTransformation}{Fourier\+Inverse\+Transformation}\+: from frequency domain to time domain based on inverse fast Fourier transform. This approach generates a complete time series all at once. 
\item \hyperlink{classPecos_1_1KarhunenLoeveInverseTransformation}{Karhunen\+Loeve\+Inverse\+Transformation}\+: from frequency domain (P\+SD) to time domain based on Karhunen-\/\+Loeve. This approach generates a complete time series all at once. 
\item \hyperlink{classPecos_1_1SamplingInverseTransformation}{Sampling\+Inverse\+Transformation}\+: from frequency domain (P\+SD) to time domain based on random sampling. This approach supports incremental generation of time series. This approach will be adapted from the Salinas implementation by Richard Koteras. 
\end{DoxyItemize}
\end{DoxyItemize}\hypertarget{index_DevProbTrans}{}\subsection{Probability Transformations}\label{index_DevProbTrans}
Class hierarchy\+: \hyperlink{classPecos_1_1ProbabilityTransformation}{Probability\+Transformation}.

Probability transformations provide a mechanism for modifying an input probability space to have desired modeling characteristics. In particular, by transforming from a given type of probability distribution to a standardized probability distribution, algorithms designed for these standard probability distributions can be applied. Specific transformations include\+:


\begin{DoxyItemize}
\item \hyperlink{classPecos_1_1NatafTransformation}{Nataf\+Transformation}\+: a nonlinear variable transformation from original, correlated probability distributions to uncorrelated standardized probability distributions. The transformation is based on C\+DF equivalence and application of the Cholesky factor of a modified correlation matrix. Nataf is currently operational and in use by D\+A\+K\+O\+TA.


\item \hyperlink{classPecos_1_1RosenblattTransformation}{Rosenblatt\+Transformation}\+: another nonlinear variable transformation from original, correlated probability distributions to uncorrelated standardized probability distributions. Whereas Nataf operates on C\+DF marginal distributions, Rosenblatt operates on joint distributions and is the preferred approach when this level of data is known. Rosenblatt is not yet supported but may be added in time. 
\end{DoxyItemize}\hypertarget{index_DevDrivers}{}\subsection{Integration Drivers}\label{index_DevDrivers}
Class hierarchy\+: \hyperlink{classPecos_1_1IntegrationDriver}{Integration\+Driver}.

This hierarchy provides a set of drivers for multivariate numerical integration for computing coefficients of a multivariate \hyperlink{classPecos_1_1BasisApproximation}{Basis\+Approximation}. Specific drivers include\+: 
\begin{DoxyItemize}
\item \hyperlink{classPecos_1_1CubatureDriver}{Cubature\+Driver}\+: Stroud cubature rules and extensions 
\item \hyperlink{classPecos_1_1SparseGridDriver}{Sparse\+Grid\+Driver}\+: Smolyak sparse grids; isotropic, anisotropic, and generalized. Sub-\/classes include\+: 
\begin{DoxyItemize}
\item \hyperlink{classPecos_1_1CombinedSparseGridDriver}{Combined\+Sparse\+Grid\+Driver} for the \char`\"{}combinatorial\char`\"{} sparse grid formulation (combines repeated tensor grids), and 
\item \hyperlink{classPecos_1_1HierarchSparseGridDriver}{Hierarch\+Sparse\+Grid\+Driver} for the hierarchical grid formulation (employs differences between consecutive quadrature levels). 
\end{DoxyItemize}
\item \hyperlink{classPecos_1_1TensorProductDriver}{Tensor\+Product\+Driver}\+: tensor-\/product quadrature; isotropic and anisotropic


\end{DoxyItemize}

\hyperlink{classPecos_1_1LHSDriver}{L\+H\+S\+Driver} is currently separate from this hierarchy.\hypertarget{index_DevNDBasis}{}\subsection{Multivariate Basis Approximations}\label{index_DevNDBasis}
Class hierarchy\+: \hyperlink{classPecos_1_1BasisApproximation}{Basis\+Approximation}.

This hierarchy provides sets of multivariate basis functions for spatial and/or time domains in support of \hyperlink{classPecos_1_1DataTransformation}{Data\+Transformation} capabilities. Specific basis function sets may include\+:


\begin{DoxyItemize}
\item Fourier\+Basis\+: Fourier basis functions which map frequencies to the time domain. 
\item K\+L\+Basis\+: discrete Karhunen-\/\+Loeve eigenfunctions from singular value decomposition of covariance matrices. 
\item \hyperlink{classPecos_1_1PolynomialApproximation}{Polynomial\+Approximation}\+: 
\begin{DoxyItemize}
\item \hyperlink{classPecos_1_1OrthogPolyApproximation}{Orthog\+Poly\+Approximation}\+: a multivariate orthogonal basis defined from one-\/dimensional orthogonal polynomials (see \hyperlink{classPecos_1_1OrthogonalPolynomial}{Orthogonal\+Polynomial} below) and a multi-\/index. 
\item \hyperlink{classPecos_1_1InterpPolyApproximation}{Interp\+Poly\+Approximation}\+: a multivariate interpolation basis defined from one-\/dimensional interpolation polynomials (local or global, value-\/based or gradient-\/enhanced; see \hyperlink{classPecos_1_1InterpolationPolynomial}{Interpolation\+Polynomial} below) and a multi-\/index. Sub-\/classes include\+: 
\begin{DoxyItemize}
\item \hyperlink{classPecos_1_1HierarchInterpPolyApproximation}{Hierarch\+Interp\+Poly\+Approximation} for multivariate interpolation using hierarchical surpluses, and 
\item \hyperlink{classPecos_1_1NodalInterpPolyApproximation}{Nodal\+Interp\+Poly\+Approximation} for multivariate interpolation using nodal values. 
\end{DoxyItemize}
\end{DoxyItemize}
\end{DoxyItemize}\hypertarget{index_Dev1DBasis}{}\subsection{Univariate Basis Approximations}\label{index_Dev1DBasis}
Class hierarchy\+: \hyperlink{classPecos_1_1BasisPolynomial}{Basis\+Polynomial}.

This hierarchy provides sets of univariate basis functions in support of the multivariate \hyperlink{classPecos_1_1BasisApproximation}{Basis\+Approximation} capabilities. Specific univariate basis functions options include\+: 
\begin{DoxyItemize}
\item \hyperlink{classPecos_1_1OrthogonalPolynomial}{Orthogonal\+Polynomial}\+: polynomial basis functions which are orthogonal with respect to a variety of continuous probability distribution function weightings\+: 
\begin{DoxyItemize}
\item \hyperlink{classPecos_1_1HermiteOrthogPolynomial}{Hermite\+Orthog\+Polynomial} 
\item \hyperlink{classPecos_1_1LegendreOrthogPolynomial}{Legendre\+Orthog\+Polynomial} 
\item \hyperlink{classPecos_1_1LaguerreOrthogPolynomial}{Laguerre\+Orthog\+Polynomial} 
\item \hyperlink{classPecos_1_1JacobiOrthogPolynomial}{Jacobi\+Orthog\+Polynomial} 
\item \hyperlink{classPecos_1_1GenLaguerreOrthogPolynomial}{Gen\+Laguerre\+Orthog\+Polynomial} 
\item \hyperlink{classPecos_1_1ChebyshevOrthogPolynomial}{Chebyshev\+Orthog\+Polynomial} 
\item \hyperlink{classPecos_1_1NumericGenOrthogPolynomial}{Numeric\+Gen\+Orthog\+Polynomial} 
\end{DoxyItemize}
\item \hyperlink{classPecos_1_1InterpolationPolynomial}{Interpolation\+Polynomial}\+: one-\/dimensional interpolation polynomials. 
\begin{DoxyItemize}
\item \hyperlink{classPecos_1_1LagrangeInterpPolynomial}{Lagrange\+Interp\+Polynomial}\+: global value-\/based Lagrange interpolation polynomials 
\item \hyperlink{classPecos_1_1HermiteInterpPolynomial}{Hermite\+Interp\+Polynomial}\+: global gradient-\/enhanced Hermite interpolation polynomials 
\item \hyperlink{classPecos_1_1PiecewiseInterpPolynomial}{Piecewise\+Interp\+Polynomial}\+: local value-\/based (linear) and gradient-\/enhanced (cubic) spline polynomials using a nodal basis formulation


\end{DoxyItemize}
\end{DoxyItemize}\hypertarget{index_DevServices}{}\section{Services}\label{index_DevServices}
A variety of services are provided in P\+E\+C\+OS for parallel computing, memory management, etc. An overview of the classes and member functions involved in performing these services is included below.


\begin{DoxyItemize}
\item Memory management\+: P\+E\+C\+OS employs the techniques of reference counting and representation sharing through the use of letter-\/envelope and handle-\/body idioms (Coplien, \char`\"{}\+Advanced C++\char`\"{}). 
\item Fast Fourier transform (F\+FT)\+: Two F\+FT libraries will be supported. First, F\+F\+TW provides platform tuned and parallel F\+FT operations in C under a G\+NU G\+PL license. Second, D\+F\+F\+T\+P\+A\+CK provides a reference F\+FT implementation in F77 in the public domain. 
\end{DoxyItemize}\hypertarget{index_DevCodeEx}{}\section{Code Examples}\label{index_DevCodeEx}
A number of test drivers are available in {\ttfamily pecos/test} that demonstrate interaction with P\+E\+C\+OS capabilities. The following drivers demonstrate basis stochastic process sample generation from power spectral density (P\+SD)\+: 
\begin{DoxyItemize}
\item pecos\+\_\+ifft\+\_\+g.\+cpp\+: scalar stochastic process generation from P\+SD using inverse F\+FT in the method of Grigoriu. 
\item pecos\+\_\+ifft\+\_\+sd.\+cpp\+: scalar stochastic process generation from P\+SD using inverse F\+FT in the method of Shinozuka and Deodatis. 
\end{DoxyItemize}

The following drivers demonstrate generation of grids for collocation methods\+: 
\begin{DoxyItemize}
\item pecos\+\_\+int\+\_\+driver.\+cpp\+: generation of integration driver sample sets (Smolyak sparse grids, tensor-\/product quadrature, cubature, and local refinable grids) 
\item pecos\+\_\+lhs\+\_\+driver.\+cpp\+: generation of an L\+HS sample set


\end{DoxyItemize}

Finally, the following drivers demonstrate miscellaneous statistical utilities\+: 
\begin{DoxyItemize}
\item boost\+\_\+test\+\_\+dist.\+cpp\+: tests Boost statistical distribution functions. 
\item boost\+\_\+test\+\_\+rng.\+cpp\+: tests Boost random number generators 
\end{DoxyItemize}\hypertarget{index_DevAddtnl}{}\section{Additional Resources}\label{index_DevAddtnl}
Additional development resources include\+:

\begin{DoxyItemize}
\item Project web pages will be maintained in the future at \href{http://dakota.sandia.gov/packages/pecos}{\tt http\+://dakota.\+sandia.\+gov/packages/pecos} \end{DoxyItemize}
